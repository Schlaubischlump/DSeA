%Thema im Praktikum soll dynamische Programmierung werden
\chapter{Vorlesung}
\section{Minimal aufspannende Bäume MST}	%Werden 2 Algorithmen machen
\paragraph{Eingabe}
\[ G=(V,E)~~E~\text{ungerichtet}~~(u,v)\in E \Rightarrow (v,u)\in E\text{ mögliche Notation } \{ u,v \} \]
$ w:E\rightarrow \mathbb{R}$
\paragraph{Gesucht}
\[ \text{Baum }T \subseteq E\]
\[ G_T=(V,T) \text{zusammenhängend (zykelfrei)} \]
\[ w(T) = \sum_{e\in T} w(e) \text{minimal} \]
\begin{figure}
\centering
\begin{subfigure}[h]{0.3\textwidth}
	\includegraphics[width=\linewidth]{19/Grafik/Spannbaum}
	\caption{Beispiel für einen Spannbaum in einem Graphen}
	\label{fig:Spannbaum}
\end{subfigure}
\begin{subfigure}[h]{0.3\textwidth}
	\includegraphics[width=\linewidth]{19/Grafik/SpannbaumBeispiel2}
	\caption{Beispiel für einen Spannbaum}
	\label{fig:Spannbaum2}
\end{subfigure}
\end{figure}

%Bild1 und 2
\paragraph{Frage} $|T|= $?
\paragraph{Antwort} $|T| = |V|-1$
\subsection{Greedy-Algorithmen zur Lösung des MST-Problems:}
Starte mit $T=\emptyset$, nehme sukzessive Kanten zu $T$ hinzu, so dass nach $|V|-1$ Schritten der gesuchte MST entstanden ist. 
Dabei benötigen wir ein Kriterium, das sicherstellt, dass gewählte Kanten zur Gesamtlösung dazugehören.
\subsection{Schnitt-Lemma:}
Betrachte eine Aufteilung (Schnitt) der Knotenmenge $V$ in $V$ und $\overline{S} = V\setminus S$ \\und Kanten $(u,v) \in E \cap S\times \overline{S}$\\
Sei $e \in E \cap S \times \overline{S}$ mit $w(e) \leq w(e') ~\forall ~e' \in E \cap S\times \overline{S}$ dann gibt es einen MST mit $e \in$ MST
\subsection{Beweis für das Schnitt-Lemma}
\begin{figure}[h]
\centering
\includegraphics[width=0.5\linewidth]{19/Grafik/SpannbaumBeweis}
\caption{}
\label{fig:SpannbaumBeweis}
\end{figure}

Sei e eine "`sichere"' Kante aus dem Schnitt-Lemma.\\
o.B.d.A. $u\in S$ und $v \in \overline{S}$.\\
Es gibt eine Zykel in $T\cup \{e\}$ und darin eine Kante $e'\in S\times \overline{S}$ mit $w(e') \geq w(e)$.\\
Ersetze $T'=T\cup\{e\}\setminus\{ e' \}$\\
$w(T') \leq w(T) \Rightarrow w(T') = w(T)$ weil $T$ ein MST.
\begin{flushright}
	q.e.d.
\end{flushright}
