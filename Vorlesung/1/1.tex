\part{Sortieren}
\chapter{Vorlesung 1}
\section{Bubblesort}
%Grafik
\subsection{Pseudocode}
\begin{lstlisting}
void bubblesort (int[] a) {
  int n = a.length;
  for (int i = 1; i < n; i++) {
    for (int j = 0; j < n-i; j++) {
      if ( a[j] < a[j+1])
        swap (a, j, j+1);
    }
  }
}
\end{lstlisting}
\paragraph{Schleifen-Invariante:} Nach dem Ablauf der i-ten Phase gilt:
\begin{center}
	Die Feldpositionen n-i,\ldots,n-i enthalten die korrekt sortierten Feldelemente
\end{center}
\paragraph{Beweis} durch Induktion nach i $\overset{i=n-1}{\Longrightarrow}$ Sortierung am Ende korrekt.
\subsection{Laufzeitanalyse}
\begin{tabular}{rcc}
	1.&Phase & n-1 \\
	2.&Phase & n-1 \\
	3.&Phase & n-1 \\
	 & $\vdots$ &  \\
	i.& Phase & n-1 \\
	 & $\vdots$ &  \\
	(n-1).&Phase & n-1 \\ \hline
	\multicolumn{3}{c}{$1+2+3+\ldots+(n+1)$}
\end{tabular}
\[ T(n)=\sum_{i=1}^{n-1} i = \frac{n(n-1)}{2}\in O(n^2) \]
\begin{tabular}{c|c}
	$n$ & $T_{real}$ \\ \hline
	$2^{10}$ & 8ms \\
	$2^{11}$ & 11ms \\
	$2^{12}$ & 26ms \\
	$\vdots$ &  \\
	$2^{16}$ & 5,819s \\
	$2^{17}$ & 23,381s \\
	$\vdots$ &  \\
	$2^{20}$ & 16min \\
	$\vdots$ &  \\
	$2^{26}$ & 52d 
\end{tabular}
\[ T_{real}(n)\approx cn^2~~ c\approx10^{-6}\]
\section{Heapsort}
\paragraph{z.B.} \begin{tabular}{ccccccccccccccc}
	21&6&4&7&12&5&3&11&14&17&19&8&9&10&42
\end{tabular}
\paragraph{Skizze}
%Grafik
\subsection{Heap-Eigenschaft}
%Grafik