\chapter{Vorlesung}

\section{Akra-Brazzi}

\begin{flalign*}
&T(n) = a \cdot T(\frac{n}{b}) + g(n)&\\
&T(1) = c&\\
&T(n) = \Theta (n^{\alpha} (1 + \int_1^n \frac{g(x)}{x^{1+\infty}} dx)) ~~~\text{mit}~\frac{a}{b^{\alpha}} = 1~~~\alpha = \log_b(a)&\\
&\text{z.B. }T(n) = 2+ \frac{n}{2} + \log(n)&
\end{flalign*}

\subsection*{Beweisidee}
\begin{flalign*}
&T(\frac{n}{b}) = a T(\frac{n}{b^2}) + g(\frac{n}{b})&\\
&T(n) = a (aT(\frac{n}{b^2}) + g(\frac{n}{b})) + g(n) = a^2 + \frac{n}{b^2} + a^1g(\frac{n}{b^1}) + a^0g(\frac{n}{b^0})&\\
&\Rightarrow a^i T(\frac{n}{b^i}) + \sum_{j=0}^{i-1} a^i g(\frac{n}{b^2})~~~\text{Rekursionsende für r = } \log_b(b)&\\
&\Theta(a^{\log_b(n)}) = \Theta(n^{\alpha})&\\
&\sum_{j=0}^{\log(n)-1} a^j g(\frac{n}{b^{\alpha}}) \approx \int_0^{\log_b(n)} a^j g(\frac{n}{b^j} dj&
\end{flalign*}

\begin{mdframed}
\textbf{Substitution}
\begin{flalign*}
&x=\frac{n}{b^j} = n \cdot b^{-j} = n \cdot e^{-j \ln(b)}& \hfill  &\frac{dx}{d_j} = n(-\ln(b))e^{-j \ln(b)} = -n \ln(b) b^j = -ln(b) x&\\
&\Rightarrow d_j = \frac{1}{-\ln(b)x} dx& \\
&a^j = b^{\log_b(a) j} = b^{\alpha j}  = (b^j)^{\alpha} = (\frac{n}{x})^{\alpha}& 
\end{flalign*}
\end{mdframed}

\begin{flalign*}
&=\int_n^1 (\frac{n}{x}) ^{\alpha} g(x) (\frac{1}{-\ln(b)x}) dx = \frac{n^{\alpha}}{\ln(b)} \cdot \int_1^n \frac{g(x)}{x^{1+\infty}} dx&\\
\\
&q.e.d&
\end{flalign*}

\pagebreak

\section{Lineare Rekursionsgleichungen}

\subsection{Fibonacci-Zahlen}

\begin{wrapfigure}[0]{r}{0.5\linewidth}
\vspace{20pt}
  \begin{tabular}{ l || c c c c c c c c c}
    \hline
    n & 0 & 1 & 2 & 3 & 4 & 5 & 6 & 7 & ... \\ \hline
    f(n) & 0 & 1 & 1 & 2 & 3 & 5 & 8 & 13 & ... \\
    \hline
  \end{tabular}
\caption{Fibonacci-Zahlen}
\end{wrapfigure}

\begin{flalign*}
&f_n = f_{n-1} + f_{n-2}& \\
&f_0 = 0& \\
&f_1 = 1&
\end{flalign*}
\vspace{20pt}


\subsection{Methode der erzeugenden Funktionen}
\[F(Z) = \sum_{n=0}^{\infty} f_n Z^n = f_0 \cdot Z^0 + f_1 \cdot Z^1 + \sum_{n=2}^{\infty} (f_{n-1} + f_{n-2}) \cdot Z^n \]
\[=Z+\sum_{n=2}^{\infty} f_{n-1} Z^n + \sum_{n=2}^{\infty} f_{n-2} Z^n\]
\[=Z + Z \sum_{n=2}^{\infty} f_{n-1} Z ^{n-1} + Z^2 \sum_{n=2}^{\infty} f_{n-2} Z^{n-2}\]
\[\Leftrightarrow F(Z) = Z + Z \cdot F(Z) + Z^2 \cdot F(Z) \]
\[\Leftrightarrow -Z = Z^2 F(Z) + Z F(Z) - F(Z) = F(Z)(Z^2+Z-1) \]
\[ F(Z) = - \frac{Z}{Z^2+Z+1} \]


\begin{mdframed}
\subsection{Einschub: Beispiel Reihenentwicklung}
\begin{flalign*}
&\frac{1}{1-Z} = \sum_{n=0}^{\infty} Z^n&
\end{flalign*}
\end{mdframed}

\[\Rightarrow F(Z) = -\frac{Z}{Z^2+Z+1} \]


\subsection{Nullstellen des Nennerpolynoms}

 \begin{tabular}{l @{\hspace{4em}} | l}
 $Z^2+Z = 1~~~|+(\frac{1}{2})^2$ 						& \textbf{Goldener Schnitt} \\[1ex]
$\Leftrightarrow (Z+\frac{1}{2})^2 = \frac{5}{4}$ 				& $\phi = \frac{1+\sqrt{5}}{2}$ \\[1ex]
$\Leftrightarrow Z_{1/2} = -\frac{1}{2} \pm \frac{\sqrt{5}}{2}$ 	& $\overline{\phi} = \frac{1-\sqrt{5}}{2}$ \\[1ex]
$\Rightarrow Z^2+ Z + 1 = (Z + \phi)(Z+\overline{\phi}) $		& \text{}
\end{tabular}

\pagebreak


\subsection{Partialbruchzerlegung}

\[\frac{A}{Z+\phi} + \frac{B}{Z+\overline{\phi}} = \frac{A\cdot (Z+\overline{\phi}) + B (Z+\phi)}{(Z+\phi)(Z+\overline{\phi})} \]
\[\Rightarrow AZ + BZ = -Z \Leftrightarrow A+B=1 ~~~\text{(1)} \]
\[A \overline{\phi} + B \phi = 0 \Leftrightarrow B = -\frac{A \overline{\phi}}{\phi} ~~~\text{(2)}\]
\[\text{(2) in (1)} A -\frac{A \overline{\phi}}{\phi} = -1 \Leftrightarrow A (1- \frac{\overline{\phi}}{\phi}) = -1 \]
\[\Leftrightarrow A = -\frac{1}{\sqrt{5}}\phi \]
\[\Rightarrow B = \frac{1}{\sqrt{5}} \overline{\phi} \]


\subsection{Lösung}
\[F(Z) = \frac{-Z}{Z^2+Z+1} = -\frac{1}{\sqrt{5}} \frac{\phi}{Z+\phi} + \frac{1}{\sqrt{5}} \frac{\overline{\phi}}{Z+\overline{\phi}} \]
\[=\frac{1}{\sqrt{5}} (\frac{1}{1+\frac{Z}{\phi}} - \frac{1}{1+\frac{Z}{\overline{\phi}}}) =\frac{1}{\sqrt{5}} (\frac{1}{1-\phi Z} - \frac{1}{1-\overline{\phi} Z})\]
\[=\frac{1}{\sqrt{5}} (\sum_{n=0}^{\infty} (\phi Z)^n - \sum_{n=0}^{\infty} (\overline{\phi} Z)^n) = \sum_{n=0}^{\infty} \frac{1}{\sqrt{5}} (\phi^n - \overline{\phi}^n) \cdot Z^n\]
\[f_n = \frac{1}{\sqrt{5}} (\phi^n - \overline{\phi}^n) ~~~\text{mit}~\phi = 1,681...~~~\overline{\phi} = -0,681...\]

\section{Quicksort (Divide and Conquer)}

\begin{figure}[h]
\includegraphics[width=0.6\linewidth]{5/Grafik/img1.png}
\end{figure}

$\Rightarrow $ Tausche 15 mit 10 \\
- Bewege Zeiger erneut \\
$\Rightarrow $ Tausche 5 und 24 \\
- Bewege Zeiger erneut \\
$\Rightarrow $ Tausche 16 und 2 \\
- Bewege Zeiger erneut \\
$\Rightarrow $ i wird größer als j $\Rightarrow $  Abbruch (tausche Pivotelement mit letztem Element in Teilliste 1) \\
$\Rightarrow $ es ergeben sich zwei Teillisten \\

4, 6, 3, 10, 7, 9 , 5, 2 |12| 16, 24, 42, 15 \\

\paragraph{best-case}  $T(n) = 2 T(\frac{n}{2}) + n = \Theta(n \log n) $

\paragraph{worst-case}  $T(n) = T(n-1) + n = \Theta(n^2) $
