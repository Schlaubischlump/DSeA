\chapter{Vorlesung 3}


\section{Landau-Notation}

$g, f : \mathbb{N} \rightarrow \mathbb{N}$

\subsection{\boldmath{$O(n)$}}
$g(n) \in O(f(n)) \Leftrightarrow c > 0 \land n_0 \in \mathbb{N}, \text{so dass für alle}~n \geq n_0~\text{gilt:} g(n) \leq c \cdot f(n) \Leftrightarrow lim_{n \rightarrow \infty} sup \frac{g(n)}{f(n)} < \infty$
\paragraph{Beispiel} \text{} \\
$lim_{n \rightarrow \infty} \frac{n \log_2(n)}{n^2} = lim_{n \rightarrow \infty} \frac{log_2(n)}{n} = lim_{n \rightarrow \infty} \frac{\frac{\ln(n)}{\ln(2)}}{n} \stackrel{\text{L' Hopital}}{=} lim_{n \rightarrow \infty} \frac{1}{\ln(2)} \cdot \frac{1}{n} = \frac{1}{\ln(2)} lim_{n \rightarrow \infty} \frac{1}{n} = 0$\\



\subsection{\boldmath{$\Omega(n)$}}
$g(n) \in \Omega(f(n)) \Leftrightarrow c > 0 \land n_0 \in \mathbb{N}, \text{so dass für alle}~n \geq n_0~\text{gilt:} g(n) \geq c \cdot f(n) \Leftrightarrow lim_{n \rightarrow \infty} inf \frac{g(n)}{f(n)} > 0$
\paragraph{Beispiel} $g(n) = n^p~~~~f(n)=n^q~~~~p \geq q$
\paragraph{Behauptung} $g(n) \in \Omega(f(n))$\\

$lim_{n \rightarrow \infty} \frac{n^p}{n^q} = \infty > 0$\\


\subsection{\boldmath{$\Theta(n)$}}
$g(n) \in \Theta(f(n)) \Leftrightarrow g(n) \in O(f(n)) \land g(n) \in \Omega(f(n))$
\paragraph{Beispiel} $g(n) = n^p + n^{p-1} + c \cdot n^2~~~~~f(n) = n^p$
\paragraph{Behauptung} $g(n) \in \Theta(f(n))$ \\
.... Rechnung

\subsection{\boldmath{$o(n)$}}

$g(n) \in o(f(n)) \Leftrightarrow lim_{n \rightarrow \infty} \frac{g(n)}{f(n)} = 0$
\paragraph{Beispiel} $g(n) = n \cdot \log_2(n)~~~~f(n) = n^2$\\
$lim_{n \rightarrow \infty} \frac{g(n)}{f(n)} = 0~~~\text{siehe oben}$\\
\paragraph{Erklärung} ''g ist asymptotisch gesehen vernachlässigbar gegenüber f.''

\pagebreak


\begin{mdframed}
\subsection{Notation}
Häufig wird:\\
$~~~~~~~~O(n) = O(n^2) = O(n^2 \cdot \log_2(n))$ \\
geschrieben, anstelle von:\\
$~~~~~~~~O(n) \subseteq O(n^2) \subseteq O(n^2 \cdot \log_2(n))$ \\
Missbrauch der Notation !!!\\

\end{mdframed}


\section{Mergesort (Divide and Conquer)}

\subsection{Pseudo-Code}
\lstinputlisting[language=C]{3/Code/mergesort.c}

\pagebreak

\subsection{Laufzeitanalyse}
$T(n) =$ Zahl der von Mergesort durchgeführten Elementarvergleiche $\approx$ Laufzeit \\
$T(n) = 2T(\frac{n}{2}) + n -1 \approx 2T(\frac{n}{2}) + n ~~~\text{mit} T(1) = 0$\\ 

\begin{mdframed}
\paragraph{Korrekter wäre} $T(n) = T(\lfloor \frac{n}{2}  \rfloor) + T(\lceil \frac{n}{2}  \rceil) + n -1~~~$ \hfill Für ungerade Zahlen \\

\end{mdframed}

\begin{flalign*}
&T(n) = 2 \cdot 2 T(\frac{n}{2}) + n \stackrel{\text{(1)}}{=} 2 ( 2 T(\frac{n}{4}) + \frac{n}{2}) + n ) = 4 T(\frac{n}{4}) + 2n&\\
&~~~~~~~\stackrel{\text{(2)}}{=} 4 \cdot (2T(\frac{n}{8}) + \frac{n}{4}) + 2n = 8T(\frac{n}{8}) + 3n = ... = 2^i \cdot T(\frac{n}{2^i})  + in&\\
\\
&T(\frac{n}{2}) = 2 T(\frac{n}{4}) + \frac{n}{2}~~~~~~(1)& \\
&T(\frac{n}{4}) = 2 T(\frac{n}{8}) + \frac{n}{4}~~~~~~(2)& \\
&....&\\
&T(1) = 0&
\end{flalign*}


\paragraph{Rekursionsende} $\frac{n}{2^i} = 1 \Leftrightarrow 2^i = n \Leftrightarrow i = \log_2(n)$\\
\begin{flalign*}
&T(n) = 2^{\log_2(n)} T(\frac{n}{2^{\log_2(n)}}) + n \log_2(n) = n T(1) + \log_2(n) = \log_2(n)&
\end{flalign*}


\paragraph{Abstraktion} \text{} \\
$T(n) =$ Laufzeit eines Divide \& Conquer Algorithmus der ein Problem dadurch löst, das es in $a$ Teilprobleme der Größe $\frac{n}{b}$ zerlegt wird, die rekursiv gelöst werden und anschließend kombiniert werden.\\

$T(n) = a\cdot T(\frac{n}{b}) + n^{\alpha} ~~~~~\alpha > 0~~~~~~\text{mit}~T(1) = 0$\\ 


\begin{flalign*}
&T(n) = a T(\frac{n}{b}) + n^{\alpha}  \stackrel{\text{(1)}}{=} a^2 T(\frac{n}{b^2}) + a(\frac{n}{b})^{\alpha} + n^{\alpha}&\\
&(1)~~~~~T(\frac{n}{b}) = a T(\frac{n}{b^2}) + (\frac{n}{b})^{\alpha}  \stackrel{\text{(2)}}{=} a^3 T(\frac{n}{b^3}) + a^2(\frac{n}{b^2})^{\alpha}  + a^1(\frac{n}{b^1})^{\alpha} +  a^0(\frac{n}{b^0})^{\alpha}&\\
&(2)~~~~~T(\frac{n}{b^2}) = a T(\frac{n}{b^3}) + (\frac{n}{b^2})^{\alpha}  = a^i T(\frac{n}{b^i}) + \sum_{j=0}^{i-1} a^j (\frac{n}{b^j})^{\alpha} =  a^i T(\frac{n}{b^i}) + n^{\alpha} \sum_{j=0}^{i-1} (\frac{a}{b^{\alpha}})^j~~\text{mit}~i= \log_b(n) \land x = \frac{a}{b^{\alpha}}&\\
&...&\\
&T(1) = 0&
\end{flalign*}

\pagebreak
