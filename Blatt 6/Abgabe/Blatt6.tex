\documentclass[a4paper,11pt,twoside]{article}
\usepackage[T1]{fontenc}
\usepackage[utf8]{inputenc}
\usepackage{ngerman, eucal, mathrsfs, amsfonts, bbm, amsmath, amssymb, stmaryrd,graphicx, array, geometry, listings, color}
\usepackage{graphicx}
\geometry{left=25mm, right=15mm, bottom=25mm}
\setlength{\parindent}{0em} 
\setlength{\headheight}{0em} 
\title{Theoretische Grundlagen der Informatik II\\ Blatt 6}
\author{Markus Vieth, David Klopp, Christian Stricker}
\date{\today}

\begin{document}

\maketitle
\cleardoublepage
\pagestyle{myheadings}
\markboth{Markus Vieth,  David Klopp, Christian Stricker}{Markus Vieth, David Klopp, Christian Stricker}

\section*{Aufgabe 1}
\underline{Anmerkung: }\\ Eine Rotation besteht aus einer konstanten Anzahl von Verknüpfungsänderungen angewendet auf eine konstante Anzahl von Knoten. Die Laufzeit liegt somit in O(1). \\
\includegraphics*[scale=0.2]{Images/Nr_1.png}

\section*{Aufgabe 2}
Füge rekursive den Wurzelknoten von $T_2$ an den größten Knoten in $T_1$. Beim wiederhochspringen der Rekursion wird die Höhe der Teilbäume verglichen (Da jeder Knoten seine Höhe kennt, reicht eine einfache Addition um die richtige Höhe des Knotens im $T_1$ zu bestimmen) und notfalls rotiert. Es muss nur ab dem größten Knoten des $T_1$ Baumes der Gesamtbaum ausgeglichen werden, da die Teilbäume des größten Knoten von $T_1$ schon ausgeglichen sind. \\
Laut Vorlesung ist der Gesamtbaum ausgeglichen, wenn der Rotier-Algorithmus rekursive einmal durchgelaufen ist. \\
Die Laufzeit ist O(log(n)), da einmal der größte Knoten in $T_1$ gesucht wird. 


\end{document}