\documentclass[a4paper,11pt,twoside]{article}
\usepackage[T1]{fontenc}
\usepackage[utf8]{inputenc}
\usepackage{ngerman, eucal, mathrsfs, amsfonts, bbm, amsmath, amssymb, stmaryrd,graphicx, array, geometry, listings, color}
\geometry{left=25mm, right=15mm, bottom=25mm}
\setlength{\parindent}{0em} 
\setlength{\headheight}{0em} 
\title{Datenstrukturen und effiziente Algorithmen\\ Blatt 4}
\author{Markus Vieth, David Klopp, Christian Stricker}
\date{\today}
\newcommand{\limesS}{\text{lim sup}}
\newcommand{\lsi}{\limesS_{n\rightarrow \infty}} %lim sup n nach inf
\newcommand{\limesinf}{\text{lim}_{n\rightarrow \infty}}


\begin{document}

\maketitle
\cleardoublepage
\pagestyle{myheadings}
\markboth{Markus Vieth,  David Klopp, Christian Stricker}{Markus Vieth, David Klopp, Christian Stricker}

\section*{Nr.1}


\section*{Nr.3}
\subsection*{a)}
\begin{tabular}{r|l}
	$x$ & Ergebnisse: \\ \hline
	$0$&$18$\\
	$1$&$15$\\
	$2$&$14$\\
	$3$&$13$\\
	$4$&$14$\\
	$5$&$15$\\
	$6$&$16$\\
	$7$&$17$\\
	$8$&$18$\\
	$9$&$19$\\
	$10$&$20$\\
	$11$&$21$\\
	$12$&$22$\\
	$13$&$23$
\end{tabular} \\

Für $x=x_2=3$, also dem Median der Menge $\{x_1, x_2, x_3\}$, wird die Summe minimal.

\pagebreak

\subsection*{b)}

Beh:\\ $\sum_{i=1}^{n} |x_i-x|$ wird minimal für $x = x_{k+1}$, d.h x := Median $\{x_1, ..., x_n\}$\\

Bew: \\
Sei die Menge aller $x_i$ sortiert. Teile die Summe in zwei gleich große Teilsummen, wobei Summe 1 kleiner als $x_{k+1}$ und Summe 2 größer als $x_{k+1} $ ist:
\[\sum_{i=1}^{2k+1}|x_i-x|\]
\[= \sum_{i=1}^{k}|x_i-x| + \sum_{i=k+2}^{2k+1}|x_i-x| + |x_{k+1}-x|\]
\[= kx - \sum_{i=1}^{k}x_i -kx+ \sum_{i=k+2}^{2k+1}x_i + |x_{k+1}-x|\]
\[= \sum_{i=k+2}^{2k+1}x_i - \sum_{i=1}^{k}x_i + |x_{k+1}-x|\]\\

\underline{Für $x=x_{k+1}$}:
\[= \sum_{i=k+2}^{2k+1}x_i - \sum_{i=1}^{k}x_i \] \\

\underline{Für $x \neq x_{k+1}$:}\\
Der Term  $|x_{k+1}-x|$ wird immer > 0 und somit größer als der Fall $x=x_k$, d.h  $\sum_{i=1}^{n} |x_i-x|$ ist genau dann minimal, wenn $x = x_{k+1}$ ist.

\subsection*{c)}

Beh:\\ $\sum_{i=1}^{n} (x_i-x)^2$ wird minimal für $x = x_{k+1}$, d.h x := Median $\{x_1, ..., x_n\}$\\

Bew: \\
\[f(x) = \sum_{i=1}^{n} (x_i -x)^2\]
\[f'(x) = -\sum_{i=1}^{n} 2(x_i -x)\]
\[f''(x) = \sum_{i=1}^{n} 2\]

\underline{$f'(x)=0:$}\\
\[-\sum_{i=1}^{n}2(x_i -x) = 0\]
\[\Leftrightarrow -2nx + \sum_{i=1}^{n}2x_i = 0\]
\[\Leftrightarrow 2\sum_{i=1}^{n}x_i = 2nx\]
\[\Leftrightarrow \frac{1}{n} \cdot \sum_{i=1}^{n}x_i = x\]\\

$x$ ist offensichtlich der Median. Da $f''(x)$ immer größer als Null ist, handelt es sich bei $x=x_{k+1}$ um ein Minimum.\\
$q.e.d$

\end{document}