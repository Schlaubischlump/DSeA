\documentclass[a4paper,11pt,twoside]{article}
\usepackage{listings}
\usepackage[T1]{fontenc}
\usepackage[utf8]{inputenc}
\usepackage{ngerman, eucal, mathrsfs, amsfonts, bbm, amsmath, amssymb, stmaryrd,graphicx, array, geometry, listings, color}
\geometry{left=25mm, right=15mm, bottom=25mm}
\setlength{\parindent}{0em} 
\setlength{\headheight}{0em}
\lstset{language=java,
					 texcl=true,
           basicstyle=\ttfamily,
           keywordstyle=\color{blue}\ttfamily,
           stringstyle=\color{red}\ttfamily,
           commentstyle=\color{green}\ttfamily,
					 breaklines=true,
					 inputencoding=utf8,
          }
\title{Datenstrukturen und effiziente Algorithmen\\ Blatt 4}
\author{Markus Vieth, David Klopp, Christian Stricker}
\date{\today}
\newcommand{\limesS}{\text{lim sup}}
\newcommand{\lsi}{\limesS_{n\rightarrow \infty}} %lim sup n nach inf
\newcommand{\limesinf}{\text{lim}_{n\rightarrow \infty}}


\begin{document}

\maketitle
\cleardoublepage
\pagestyle{myheadings}
\markboth{Markus Vieth,  David Klopp, Christian Stricker}{Markus Vieth, David Klopp, Christian Stricker}

\section*{Nr.1}
\subsection*{Erklärung:}
Eingabe: Integer k für Anzahl an Farben, Integer l für Anzahl an Muster. Ein 2x2n Array. In der ersten Zeile stehen die Farben in Integer codiert, in der zweiten Zeile analog für die Muster(bei 5 verschiedenen Muster/Farben gibt es die Integerwerte 0-4, jeder Wert steht für eine Farbe).\\
Algorithmus: Man iterriert über das Eingabearray und speichert die Anzahl der jeweiligen Farben in ein Array. Dann wird nochmal über das Array iterriert, wenn an der i-ten Stelle, die Socke mit falsche Farbe n steht, wird diese mit einer anderen Socke im n Bereich vertauscht.\\
Danach wird für jeden Farbbereich analog zum Farbensortieren nach Muster sortiert(Anzahl der Muster bestimmen, an die richtige Stelle tauschen).\\
Damit kommt eine Laufzeit von O(n+k+k*l). Im Worst Case, wenn k=l=n wäre, beträgt die Laufzeit O($n^2$).\\
O(4*n), da man 2 mal über das ganze Array iteririert, um die Anzahl (an Farben/Muster) zu bestimmen\\
O(8*n), da höchstens n-mal vertauscht wird, 2n Zugriffe zum Zwischenspeichern, 4n Zugriffe zum Überschreiben und 2n Zugriffe für die Zwischenspeicherungen wieder zu speichern.\\
Analog beim Mustertauschen.\\
Der Platzverbrauch ist 4n (Für das Eingabe 2x2n Array) + 2k (für sockenarray) + 2l (musteranzahlarray) => höchstens 8n (Bei n verschiedenen Farben und n verschiedene Muster)

\subsection*{Code:}
\lstinputlisting {Suche.java}


\pagebreak

\section*{Nr.2}
\subsection*{a)}
Überlegung analog zur Vorlesung:\\
\paragraph*{m=3}
\[\exists~\frac{2n}{6}\text{ Elemente }\leq p\Rightarrow \text{Es existieren maximal }\frac{4n}{6}\text{ Elemente }\geq p\]
\[\exists~\frac{2n}{6}\text{ Elemente }\geq p\Rightarrow \text{Es existieren maximal }\frac{4n}{6}\text{ Elemente }\leq p\]
\paragraph*{m=7}
\[\exists~\frac{4n}{14}\text{ Elemente }\leq p\Rightarrow \text{Es existieren maximal }\frac{10n}{14}\text{ Elemente }\geq p\]
\[\exists~\frac{4n}{14}\text{ Elemente }\geq p\Rightarrow \text{Es existieren maximal }\frac{10n}{14}\text{ Elemente }\leq p\]
\subsection*{b)}
Überlegung analog zur Vorlesung:\\
\paragraph*{m=3}
\[T(n)=T\left(\frac{n}{3}\right)+n+T\left(\frac{2n}{3}\right)\]
Akra-Bazzi:
\[g(n)=n,~a_1=a_2=1,~b_1=3~b_2=\frac{3}{2}\]
\[1=\left(\frac{1}{3}\right)^\alpha+\left(\frac{2}{3}\right)^\alpha\]
\[\Leftrightarrow \alpha = 1\]
\[T(n)=n\left(1+\int_{1}^{n}\frac{x}{x^{2}}dx\right)=n\left(1+\int_{1}^{n}\frac{1}{x}dx\right)=n\left(1+\left[\ln(x)\right]_1^n\right)=n+n\ln(n)\in O(n\log(n))\neq O(n)\]
Somit ist die Laufzeit für $m=3$ nicht linear.
\paragraph*{m=7}
\[T(n)=T\left(\frac{n}{7}\right)+n+T\left(\frac{5n}{7}\right)\]
\subparagraph*{Zu zeigen:}
\[\exists~c>0:T(n)\leq c\cdot n\]
\subparagraph*{Beweis:}
\[T(n)\leq c\cdot \frac{n}{7}+n+c\cdot\frac{5n}{7} \leq c\cdot n\]
\[c=7 \Leftrightarrow n+n+5n \leq 7n \Rightarrow \exists~c>0 : T(n)\leq c\cdot n\Rightarrow T(n)\in O(n)\] 
Somit ist die Laufzeit für $m=7$ linear.


\pagebreak

\section*{Nr.3}
\subsection*{a)}
\begin{tabular}{r|l}
	$x$ & Ergebnisse: \\ \hline
	$0$&$18$\\
	$1$&$15$\\
	$2$&$14$\\
	$3$&$13$\\
	$4$&$14$\\
	$5$&$15$\\
	$6$&$16$\\
	$7$&$17$\\
	$8$&$18$\\
	$9$&$19$\\
	$10$&$20$\\
	$11$&$21$\\
	$12$&$22$\\
	$13$&$23$
\end{tabular} \\

Für $x=x_2=3$, also dem Median der Menge $\{x_1, x_2, x_3\}$, wird die Summe minimal.


\subsection*{b)}

Beh:\\ $\sum_{i=1}^{n} |x_i-x|$ wird minimal für $x = x_{k+1}$, d.h x := Median $\{x_1, ..., x_n\}$\\

Bew: \\
Sei die Menge aller $x_i$ sortiert. Teile die Summe in zwei gleich große Teilsummen, wobei Summe 1 kleiner als $x_{k+1}$ und Summe 2 größer als $x_{k+1} $ ist:
\[\sum_{i=1}^{2k+1}|x_i-x|\]
\[= \sum_{i=1}^{k}|x_i-x| + \sum_{i=k+2}^{2k+1}|x_i-x| + |x_{k+1}-x|\]
\[= kx - \sum_{i=1}^{k}x_i -kx+ \sum_{i=k+2}^{2k+1}x_i + |x_{k+1}-x|\]
\[= \sum_{i=k+2}^{2k+1}x_i - \sum_{i=1}^{k}x_i + |x_{k+1}-x|\]\\

\underline{Für $x=x_{k+1}$}:
\[= \sum_{i=k+2}^{2k+1}x_i - \sum_{i=1}^{k}x_i \] \\

\underline{Für $x \neq x_{k+1}$:}\\
Der Term  $|x_{k+1}-x|$ wird immer > 0 und somit größer als der Fall $x=x_k$, d.h  $\sum_{i=1}^{n} |x_i-x|$ ist genau dann minimal, wenn $x = x_{k+1}$ ist.

\subsection*{c)}

Beh:\\ $\sum_{i=1}^{n} (x_i-x)^2$ wird minimal für $x = \frac{1}{n} \cdot \sum_{i=1}^{n}x_i$, d.h x := arithmetisches Mittel von $\{x_1, ..., x_n\}$\\

Bew: \\
\[f(x) = \sum_{i=1}^{n} (x_i -x)^2\]
\[f'(x) = -\sum_{i=1}^{n} 2(x_i -x)\]
\[f''(x) = \sum_{i=1}^{n} 2\]

\underline{$f'(x)=0:$}\\
\[-\sum_{i=1}^{n}2(x_i -x) = 0\]
\[\Leftrightarrow -2nx + \sum_{i=1}^{n}2x_i = 0\]
\[\Leftrightarrow 2\sum_{i=1}^{n}x_i = 2nx\]
\[\Leftrightarrow \frac{1}{n} \cdot \sum_{i=1}^{n}x_i = x\]\\

Da $f''(x)$ immer größer als Null ist, handelt es sich bei $x = \frac{1}{n} \cdot \sum_{i=1}^{n}x_i$ um ein Minimum.\\
$q.e.d$

\end{document}