\documentclass[a4paper,11pt,twoside]{article}
\usepackage[T1]{fontenc}
\usepackage[latin1]{inputenc}
\usepackage{ngerman, eucal, mathrsfs, amsfonts, bbm, amsmath, amssymb, stmaryrd,graphicx, array, geometry, listings, color}
\geometry{left=25mm, right=15mm, bottom=25mm}
\setlength{\parindent}{0em} 
\setlength{\headheight}{0em} 
\title{Datenstrukturen und effiziente Algorithmen\\ Blatt 1}
\author{Markus Vieth, David Klopp, Christian Stricker}
\date{\today}
\newcommand{\limesS}{\text{lim sup}}
\newcommand{\lsi}{\limesS_{n\rightarrow \infty}} %lim sup n nach inf
\newcommand{\limesinf}{\text{lim}_{n\rightarrow \infty}}


\begin{document}

\maketitle
\cleardoublepage
\pagestyle{myheadings}
\markboth{Markus Vieth,  David Klopp, Christian Stricker}{Markus Vieth, David Klopp, Christian Stricker}

\section*{Nr.1}

\subsection*{a)}
1.) Invariante:
Verschiebe jede Zahl solange nach hinten im Array, wie der Nachfolger kleiner ist als die aktuelle Zahl.
\newline
\newline
2.) Schleifenbedingung:
Solange mindestens eine Vertauschung stattgefunden hat, wiederhole die Schleife.
\newline


\subsection*{b)}
\underline{Worst-Case:} Liste in umgekehrter Reihenfolge
\newline
=> 1. Verschiebe erstes Element um n-1 Stellen
\newline
=> 2. Verschiebe zweites Element um n-2 Stellen
\newline
=> ....
\newline
Hieraus ergibt sich:
\newline 
\[\sum^{n-1}_{k=0}k = (n-1)\cdot \frac{n}{2} = \frac{n^{2}-n}{2}   \in O(n^{2})\]


\subsection*{c)}
Kleine Elemente am Ende werden automatisch an den Anfang des Arrays bewegt, wenn die gro�en Elemente aufsteigen.
\newline
=> gro�e Elemente am Anfang sind schlimmer, da diese den gesamten Array durchlaufen m�ssen.


\subsection*{d)}
Da der Algorithmus eine ''gr��er'' und keine ''gr��ergleich'' Operation durchf�hrt, ist dieser stabil.



\newpage


\section*{Nr.2}

\subsection*{a)}
\underline{I.A.:}    
\qquad{n = 0}
\[\sum^{0}_{k=1}2^{k}\cdot\left(k+1\right)=0\cdot2^{1}\]
\[\Leftrightarrow 0 = 0\]
\underline{I.B.:}    
\[\sum^{n}_{k=1}2^{k}\cdot\left(k+1\right)=n\cdot2^{n+1}\]
\underline{I.V.:}   
\newline 
\qquad{Die Behauptung gelte f�r ein festes n.}
\newline
\newline
\underline{I.S.:}   
\[\sum^{n+1}_{k=1}2^{k}\cdot\left(k+1\right)=\sum^{n}_{k=1}2^{k}\cdot\left(k+1\right) + 2^{n+1}(n+2)\]
\[= 2^{n+1}(n+n+2)\]
\[= 2\cdot2^{n+1}(n+1)\]
\[= 2^{n+2}(n+1)\]
\[q.e.d\]



\subsection*{b)}
\underline{I.A.:}    
\qquad{n = 0}
\[0=3\cdot0^{2}+0\]
\[\Leftrightarrow 0 = 0\]
\underline{I.B.:}    
\[\sum^{2n}_{k=n+1}2k=3n^{2}+n\]
\underline{I.V.:}   
\newline 
\qquad{Die Behauptung gelte f�r ein festes n.}
\newline
\newline
\underline{I.S.:}   
\[\sum^{2n+2}_{k=n+2}2k=\sum^{2n}_{k=n+1}2k-2(n+1)+2(2n+1)+2(2n+2)\]
\[=3n^{2}+n+2(3n+2)\]
\[=3(n^{2}+2n+1)+(n+1)\]
\[=3(n+1)^{2}+(n+1)\]
\[q.e.d\]


\subsection*{c)}
\underline{I.A.:}    
\qquad{Vermutung: m = 1, w�hle n = 0}
\[0=0!-1\]
\[\Leftrightarrow 0 = 1-1\]
\[\Leftrightarrow 0 = 0\]
\underline{I.B.:}    
\[\sum^{n}_{k=1}k\cdot k!=(n+1)!-1\]
\underline{I.V.:}   
\newline 
\qquad{Die Behauptung gelte f�r ein festes n.}
\newline
\newline
\underline{I.S.:}   
\[\sum^{n+1}_{k=1}k\cdot k!=\sum^{n}_{k=1}k\cdot k!+(n+1)(n+1)!\]
\[= (n+1)! \cdot (1+n+1)-1\]
\[= (n+1)! \cdot (n+2)-1\]
\[= (n+2)!-1\]
\[q.e.d\]

\section*{Nr. 3}
\subsection*{a)}
\[\log_b a \cdot \log_c b -\log_c a = \log_c b^{\log_b a}-\log_c a= \log_c a-\log_c a =0\]
\subsection*{b)}
\[\log_b \left(\frac{a}{b}\right)^c-c\cdot\log_b a=c\cdot\log_b a- c\cdot\log_b b - c\cdot\log_b a = -c \log_b b = -c \]
\subsection*{c)}
\[2\log_b \sqrt{ab}+\log_2 \frac{1}{\sqrt{a}}\log_b 4 = 2\log_b (ab)^{\frac{1}{2}}+\log_2 a^{-\frac{1}{2}}\log_b 2^2 =\log_b (ab)-\log_2 a\cdot\log_b 2\]
\[ = \log_b a + \log_b b - \log_b 2^{\log_2 a} = \log_b a +1 -\log_b a= 1  \]
\subsection*{d)}
\[\left(b^{\frac{1}{d}\cdot \log_b c}\right)^d\cdot\frac{1}{2}-\frac{1}{2}=\left(b^{\log_b c}\right)\cdot \frac{1}{c}-\frac{1}{2}=c\cdot\frac{1}{c}-\frac{1}{2}=\frac{1}{2} \]

\end{document}