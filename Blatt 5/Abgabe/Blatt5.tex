\documentclass[a4paper,11pt,twoside]{article}
\usepackage[T1]{fontenc}
\usepackage[utf8]{inputenc}
\usepackage{ngerman, eucal, mathrsfs, amsfonts, bbm, amsmath, amssymb, stmaryrd,graphicx, array, geometry, listings, color}
\usepackage{graphicx}
\geometry{left=25mm, right=15mm, bottom=25mm}
\setlength{\parindent}{0em} 
\setlength{\headheight}{0em} 
\title{Theoretische Grundlagen der Informatik II\\ Blatt 5}
\author{Markus Vieth, David Klopp, Christian Stricker}
\date{\today}

\begin{document}

\maketitle
\cleardoublepage
\pagestyle{myheadings}
\markboth{Markus Vieth,  David Klopp, Christian Stricker}{Markus Vieth, David Klopp, Christian Stricker}

\section*{Aufgabe 2}
\underline{Anmerkung: } Pfeile ohne Endknoten deuten einen beliebig großen Teilbaum an. Teilbäume bei Knoten D wurden vergessen einzuzeichnen, können aber natürlich auch existieren.
\subsection*{a)} Ersetze z mit seinem rechten Kind. Da z kein linkes Kind besitzt, ist das Löschen von z fertig.\\
\includegraphics*[scale=0.2]{Images/A.png}
\subsection*{b)}Ersetze z mit dem linken Kind. Da z kein rechtes Kind besitzt, ist das Löschen von z fertig.\\
\includegraphics*[scale=0.2]{Images/B.png}
\subsection*{c)} Hänge den linken Teilbaum von z an y und tausche y mit z. \\
\includegraphics*[scale=0.2]{Images/C.png}
\subsection*{d)} Suche das kleinste Element y vom rechten Teilbaum von z. Setze y als Parent vom rechten Kind von z und das rechte Kind von z als rechtes Kind von y. Danach ersetze z mit y und setze das linke Kind von z an die Stelle vom linken Kind von y. Das Löschen ist fertig. \\
\includegraphics*[scale=0.2]{Images/D_step_1.png} \\
\includegraphics*[scale=0.2]{Images/D_step_2.png} \\
\includegraphics*[scale=0.2]{Images/D_step_3.png}



\end{document}